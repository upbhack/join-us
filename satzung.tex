\documentclass{article}

\usepackage{verbatim}
\usepackage{german}
\usepackage[utf8]{inputenc}
\usepackage{enumerate}

\setlength{\textwidth}{15cm}
\setlength{\textheight}{24cm}
\addtolength{\topmargin}{-2cm}
\addtolength{\oddsidemargin}{-1.5cm}

\title{\textsf{\textbf{Vereinssatzung}}}
\author{}
\date{}
\hyphenation{URNIX}

\begin{document}
\maketitle

\begin{enumerate}[\textsection  1.]
\item \textsf{\textbf{Name und Sitz}}

	Der Verein führt den Namen „/upb/hack“. Er soll in das Vereinsregister eingetragen werden und trägt dann den Zusatz „e.V.“ Der Sitz des Vereins ist Paderborn.

\item \textsf{\textbf{Geschäftsjahr}}

	Das Geschäftsjahr ist das Kalenderjahr.

\item \textsf{\textbf{Zweck des Vereins}}

	Der Verein verfolgt ausschließlich und unmittelbar gemeinnützige Zwecke im Sinne des Abschnitts „Steuerbegünstigte Zwecke“ der Abgabenordnung.
	
	Zweck des Vereins ist die Förderung von Wissenschaft und Forschung, Volks- und Berufsbildung bezüglich der Sicherheit informationstechnischer Systeme.
	
	Der Satzungszweck wird verwirklicht insbesondere durch Durchführung und Besuch wissenschaftlicher
	Veranstaltungen sowie der Teilnahme an praxisbezogenen Wettkämpfen.

\item \textsf{\textbf{Selbstlose Tätigkeit}}

	Der Verein ist selbstlos tätig; er verfolgt nicht in erster Linie eigenwirtschaftliche Zwecke.

\item \textsf{\textbf{Mittelverwendung}}

	Mittel des Vereins dürfen nur für die satzungsmäßigen Zwecke verwendet werden. Die Mitglieder
	erhalten keine Zuwendungen aus Mitteln des Vereins.
	
\item \textsf{\textbf{Verbot von Begünstigungen}}

	Es darf keine Person durch Ausgaben, die dem Zweck der Körperschaft fremd sind, oder
	durch unverhältnismäßig hohe Vergütungen begünstigt werden.
	
\item \textsf{\textbf{Erwerb der Mitgliedschaft}}

	Vereinsmitglieder können natürliche Personen oder juristische Personen werden.
	Der Aufnahmeantrag ist schriftlich zu stellen;
	über den Aufnahmeantrag entscheidet der Vorstand.
	Gegen die Ablehnung, die keiner Begründung bedarf, steht dem/der Bewerber/in die
	Berufung an die Mitgliederversammlung zu, welche dann endgültig entscheidet.

\item \textsf{\textbf{Beendigung der Mitgliedschaft}}

	Die Mitgliedschaft endet durch Austritt, Ausschluss, Tod oder Auflösung der juristischen
	Person.
	
	Der Austritt erfolgt durch schriftliche Erklärung gegenüber einem vertretungsberechtigten
	Vorstandsmitglied. Die schriftliche Austrittserklärung muss mit einer Frist von
	einem Monat jeweils zum Ende des Folgemonats gegenüber dem Vorstand erklärt
	werden.
	
	Ein Ausschluss kann nur aus wichtigem Grund erfolgen. Wichtige Gründe sind insbesondere
	ein die Vereinsziele schädigendes Verhalten oder die Verletzung satzungsmäßiger
	Pflichten. Über den Ausschluss entscheidet der Vorstand. 
	Gegen den Ausschluss steht dem Mitglied die Berufung an die
	Mitgliederversammlung zu, die schriftlich binnen eines Monats an den Vorstand zu
	richten ist. Die Mitgliederversammlung entscheidet im Rahmen des Vereins endgültig.
	Dem Mitglied bleibt die Überprüfung der Maßnahme durch Anrufung der ordentlichen
	Gerichte vorbehalten. Die Anrufung eines ordentlichen Gerichts hat aufschiebende
	Wirkung bis zur Rechtskraft der gerichtlichen Entscheidung.
	
\item \textsf{\textbf{Beiträge}}

	Von den Mitgliedern werden keine Beiträge erhoben. 

\item \textsf{\textbf{Organe des Vereins}}

	Organe des Vereins sind
	die Mitgliederversammlung
	und der Vorstand.
	
\item \textsf{\textbf{Mitgliederversammlung}}

	Die Mitgliederversammlung ist das oberste Vereinsorgan. Zu ihren Aufgaben gehören insbesondere
	die Wahl und Abwahl des Vorstands, Entlastung des Vorstands, Entgegennahme der
	Berichte des Vorstandes, Wahl der Kassenprüfern/innen, Beschlussfassung über die Änderung der Satzung, Beschlussfassung über die Auflösung
	des Vereins, Entscheidung über Aufnahme und Ausschluss von Mitgliedern in Berufungsfällen
	sowie weitere Aufgaben, soweit sich diese aus der Satzung oder nach dem Gesetz
	ergeben.

	Im erstem Quartal eines jeden Geschäftsjahres findet eine ordentliche Mitgliederversammlung
	statt.

	Der Vorstand ist zur Einberufung einer außerordentlichen Mitgliederversammlung
	verpflichtet, wenn mindestens ein Drittel der Mitglieder dies schriftlich unter Angabe
	von Gründen verlangt.
	
	Die Mitgliederversammlung wird vom Vorstand unter Einhaltung einer Frist von einem
	Monat schriftlich per E-Mail unter Angabe der Tagesordnung einberufen. Die Frist beginnt mit dem
	auf die Absendung des Einladungsschreibens folgenden Tag. Das Einladungsschreiben gilt als
	den Mitgliedern zugegangen, wenn es an die letzte dem Verein bekannt gegebene E-Mail Adresse
	gerichtet war.

	Die Tagesordnung ist zu ergänzen, wenn dies ein Mitglied bis spätestens eine Woche vor dem
	angesetzten Termin schriftlich beantragt. Die Ergänzung ist zu Beginn der Versammlung bekanntzumachen.
	Anträge über die Abwahl des Vorstands, über die Änderung der Satzung und über die Auflösung
	des Vereins, die den Mitgliedern nicht bereits mit der Einladung zur Mitgliederversammlung
	zugegangen sind, können erst auf der nächsten Mitgliederversammlung beschlossen
	werden.

	Die Mitgliederversammlung ist ohne Rücksicht auf die Zahl der erschienenen Mitglieder beschlussfähig.
	Die Mitgliederversammlung wird von einem Vorstandsmitglied geleitet.
	Zu Beginn der Mitgliederversammlung ist ein Schriftführer zu wählen.
	Jedes Mitglied hat eine Stimme. Das Stimmrecht kann nur persönlich oder für ein Mitglied
	unter Vorlage einer schriftlichen Vollmacht ausgeübt werden.
	Bei Abstimmungen entscheidet die einfache Mehrheit der abgegebenen Stimmen.
	Satzungsänderungen und die Auflösung des Vereins können nur mit einer Mehrheit von 2/3
	der anwesenden Mitglieder beschlossen werden.
	Stimmenthaltungen und ungültige Stimmen bleiben außer Betracht.
	Über die Beschlüsse der Mitgliederversammlung ist ein Protokoll anzufertigen, das vom
	Versammlungsleiter und dem Schriftführer zu unterzeichnen ist.

\item \textsf{\textbf{Vorstand}}

	Der Vorstand im Sinn des §26 BGB besteht aus dem/der 1. und 2. Vorsitzenden und
	dem/der Kassierer/in. Sie vertreten den Verein gerichtlich und außergerichtlich. Zwei
	Vorstandsmitglieder vertreten gemeinsam.
	
	Der Vorstand wird von der Mitgliederversammlung auf die Dauer von einem Jahr gewählt.
	Vorstandsmitglieder können nur Mitglieder des Vereins werden.
	Wiederwahl ist zulässig.
	
	Der Vorstand bleibt solange im Amt, bis ein neuer Vorstand gewählt ist.
	Bei Beendigung der Mitgliedschaft im Verein endet auch das Amt als Vorstand.

\item \textsf{\textbf{Kassenprüfung}}

	Die Mitgliederversammlung wählt für die Dauer von einem Jahr eine/n Kassenprüfer/in.
	Diese/r darf nicht Mitglied des Vorstands sein.
	Wiederwahl ist zulässig.
	
\newpage

\item \textsf{\textbf{Inventar}}

	Es wird eine Inventurliste über Gegenstände in der Verfügungshoheit des Vereins gefürt. Jedes Mitglied verplichtet sich diese Liste zu aktualisieren, falls Gegenstände entnommen oder hinzugefügt werden. 
	
	Die Mitgliederversammlung  wählt für die Dauer von einem Jahr einen Materialwart. Diese/r darf nicht Mitglied des Vorstands sein. Wiederwahl ist zulässig. 
	Der Materialwart verpflichtet sich jedes halbe Jahr eine Inventur durchzufüren und die Liste auf vollständigkeit, bzw. korrektheit zu überprüfen.
	

\item \textsf{\textbf{Vereinsräumlichkeiten}}

	Der Vorstand, der/die Kassenprüfer/in und der Materialwart sind stehts zum Betreten der Vereinsräumlichkeiten befugt. Darüber hinaus kann der Vorstand eine vereinsinterne Liste mit Personen führen die zum freien Betreten der Vereinsräume befugt sind.

\item \textsf{\textbf{Auflösung des Vereins}}

	Bei Auflösung oder Aufhebung des Vereins oder bei Wegfall steuerbegünstigter Zwecke fällt
	das Vermögen des Vereins an Medien und Technik für Kinder und Jugend e.V., der es unmittelbar und ausschließlich
	für gemeinnützige oder mildtätige Zwecke zu verwenden hat. 
	%Beispiel f¨ur einen gemeinnützigen Verein.
	
	
	\begin{comment}
	Anmerkung:
	In einer Vereinssatzung müssen als wesentlicher Bestandteil enthalten sein (in der Mustersatzung
	durch fette Schrift hervorgehoben):
	Bestimmungen über den Namen, Sitz und Zweck des Vereins und darüber, dass er in
	das Vereinsregister eingetragen werden soll (in der Mustersatzung: § 1, § 3 zweiter
	Gliederungspunkt),
	Bestimmungen über Eintritt und Austritt der Mitglieder (in der Mustersatzung: §§ 7, 8),
	Bestimmungen darüber, ob und welche Beiträge die Mitglieder zu leisten haben (in der
	Mustersatzung: § 9),
	Bestimmungen über die Bildung des vertretungsberechtigten Vorstandes (in der Mustersatzung:
	§ 12 erster Gliederungspunkt),
	Bestimmungen über die Voraussetzungen, unter denen eine Mitgliederversammlung
	einzuberufen ist, über die Form der Einberufung und über die Beurkundung der Beschlüsse
	der Mitgliederversammlung (in der Mustersatzung: § 11 zweiter und dritter
	Gliederungspunkt, vierter Gliederungspunkt Satz 1, letzter Gliederungspunkt),
	das Datum der Errichtung.
	5
	Die Satzung eines gemeinnützigen Vereins muss aus steuerrechtlichen Gründen auch die in
	kursiver Schrift wiedergegebenen Festlegungen der Mustersatzung (§§ 3, 4, 5, 6, 14) enthalten
	(§ 60 Abs. 1 der Abgabenordnung).
	\end{comment}
	
\end{enumerate}
\vfill
\begin{minipage}{5cm}
\hrule\hfill
\newline
\vspace{0.5cm}
\centering Ort, Datum
\end{minipage}

\end{document}